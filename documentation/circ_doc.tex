\documentclass{article}
\usepackage{graphicx,tikz}
\usepackage{tabu}
\usepackage{caption}
\usepackage{subcaption}
\usepackage{hyperref}

\newcommand\bred[1]{\textcolor{red}{\textbf{#1}}}
\begin{document}

\title{Benchmark Circuits for IBM's Quantum Computer}
\date{}
\maketitle

\section{Introduction}
IBM's 5 qubit quantum computer \cite{IBMQ} supports gates from the Clifford+T gate library. 
This repository contains some  Clifford+T circuits that have been transformed to be executed on IBM's Q5.

\section{Benchmark Circuits}

The following circuits are available in the folder labeled {\tt original}.
\vspace{5mm}

\begin{tabular}{|l|r|r|r|r|c|}
   \hline
   Name & Qubits & Gates & Depth & T-depth & Source \\ \hline  \hline
   {\tt 01.qc} & 5 & 51 & 28 & 9 & \cite{Qbench}  \\  \hline
   
   {\tt 1.qc} & 3 & 17 & 11 & 6 & \cite{Qbench}  \\  \hline
   
    {\tt 3\_17\_b.qc} & 3 & 33 & 23 & 5 & \cite{DBLP:conf/rc/MillerSD14}  \\  \hline
    {\tt 3\_17\_c.qc} & 3 & 35 & 26 & 6 & \cite{DBLP:conf/rc/MillerSD14}  \\  \hline
    {\tt 3\_17\_d.qc} & 3 & 35 & 24 & 4 & \cite{DBLP:conf/rc/MillerSD14}  \\  \hline
    {\tt 3\_17\_e.qc} & 3 & 33 & 21 & 4 & \cite{DBLP:conf/rc/MillerSD14}  \\  \hline
    
    {\tt 17.qc} & 4 & 43 & 30 & 4 & \cite{Qbench}  \\  \hline
    
    {\tt a2x\_c.qc} & 4 & 31 & 22 & 5 & \cite{DBLP:conf/rc/MillerSD14}  \\  \hline
    {\tt a2x\_e.qc} & 4 & 30 & 20 & 4 & \cite{DBLP:conf/rc/MillerSD14}  \\  \hline
    
    {\tt a3x\_c.qc} & 5 & 48 & 37 & 9 & \cite{DBLP:conf/rc/MillerSD14}  \\  \hline
    {\tt a3x\_c.qc} & 5 & 44 & 33 & 8 & \cite{DBLP:conf/rc/MillerSD14}  \\  \hline
    
   {\tt Full\_Adder\_c.qc} & 4 & 20 & 19 & 7 & \cite{DBLP:conf/rc/MillerSD14}  \\  \hline
   {\tt Full\_Adder\_d.qc} & 4 & 22 & 15 & 2 & \cite{DBLP:conf/rc/MillerSD14}  \\  \hline
   {\tt Full\_Adder\_e.qc} & 4 & 21 & 12 & 2 & \cite{DBLP:conf/rc/MillerSD14}  \\  \hline
   
   {\tt Toffoli\_c.qc} & 3 & 17 & 16 & 6 & \cite{DBLP:conf/rc/MillerSD14}  \\  \hline
   {\tt Toffoli\_d.qc} & 3 & 17 & 12 & 3 & \cite{DBLP:conf/rc/MillerSD14}  \\  \hline
   {\tt Toffoli\_e.qc} & 3 & 17 & 12 & 3 & \cite{DBLP:conf/rc/MillerSD14}  \\  \hline
  \end{tabular} 
  \vspace{5mm}
  
  The transformed circuits---to fit the Q5 architecture---are found in the folder labeled {\tt IBM}.
  Different permutations, produce different results.
  Since the computer has 5 available qubits, circuits can be extended to 5 qubits at no cost.
  The names of the circuits are obtained by taken the original name and appending the permutation to it.
  A summary is given below.
  
  \vspace{5mm}
  \begin{tabu}{|l|r|r|r|r|}
   \hline
   Name & Qubits & Gates & Depth & T-depth  \\ \hline  \hline
  {\tt 01\_01234.qc} & 5 & 149 &  &   \\  \hline
  {\tt 01\_01342.qc} & 5 & \bred{77} &  &   \\  \hline
  \tabucline[2pt]{-}
  
  {\tt 1\_01234.qc} & 5 & 29 &  &   \\  \hline
   {\tt 1\_02134.qc} & 5 & \bred{25} &  &   \\  \hline
  \tabucline[2pt]{-}
  
   {\tt 3\_17\_b\_01234.qc} & 5 & 49 &  &   \\  \hline
   {\tt 3\_17\_b\_02134.qc} & 5 & \bred{43} &  &   \\  \hline
   {\tt 3\_17\_c\_01234.qc} & 5 & 49 &  &   \\  \hline
   {\tt 3\_17\_c\_02134.qc} & 5 & \bred{43} &  &   \\  \hline
   {\tt 3\_17\_d\_01234.qc} & 5 & 51 &  &   \\  \hline
   {\tt 3\_17\_d\_02134.qc} & 5 & 47 &  &   \\  \hline
   {\tt 3\_17\_e\_01234.qc} & 5 & 49 &  &   \\  \hline
   {\tt 3\_17\_e\_02134.qc} & 5 & \bred{43} &  &   \\  \hline
   \tabucline[2pt]{-}
   
    {\tt 17\_01234.qc} & 5 & 141 &  &  \\  \hline
    {\tt 17\_03421.qc} & 5 & \bred{119} &  &  \\  \hline
    \tabucline[2pt]{-}
    
   {\tt a2x\_c\_01234.qc} & 5 & 87 &  &  \\  \hline
   {\tt a2x\_c\_02341.qc} & 5 & 59 &  &  \\  \hline
   {\tt a2x\_e\_01234.qc} & 5 & 70 &  &  \\  \hline
   {\tt a2x\_e\_02341.qc} & 5 & \bred{52} &  &  \\ 
   \tabucline[2pt]{-}
   
   {\tt a3x\_c\_01234.qc} & 5 & 176 &  &  \\  \hline
   {\tt a3x\_c\_10324.qc} & 5 & 86 &  &  \\  \hline
   {\tt a3x\_d\_01234.qc} & 5 & 156 &  &  \\  \hline
   {\tt a3x\_d\_01324.qc} & 5 & \bred{66} &  &  \\  \hline
    \tabucline[2pt]{-}
    
   {\tt Full\_Adder\_c\_01234.qc} & 5 & 60 &  &    \\  \hline
   {\tt Full\_Adder\_c\_01324.qc} & 5 & \bred{28} &  &    \\  \hline
   {\tt Full\_Adder\_d\_01234.qc} & 5 & 74 &  &    \\  \hline
   {\tt Full\_Adder\_d\_01324.qc} & 5 & 42 &  &    \\  \hline
   {\tt Full\_Adder\_e\_01234.qc} & 5 & 55 &  &    \\  \hline
   {\tt Full\_Adder\_e\_01324.qc} & 5 & 37 &  &    \\ 
    \tabucline[2pt]{-}
   
   {\tt Toffoli\_c\_01234.qc} & 5 & \bred{17} &  &   \\  \hline
   {\tt Toffoli\_d\_01234.qc} & 5 & 25 &  &  \\  \hline
   {\tt Toffoli\_e\_01234.qc} & 5 & 23 &  &   \\  \hline

  \end{tabu} 
  \vspace{5mm}

The same circuits are available (in IBM format) in the folder labelled {\tt qasm}.
\bibliographystyle{plain}
\bibliography{doc_lit} 

\end{document}